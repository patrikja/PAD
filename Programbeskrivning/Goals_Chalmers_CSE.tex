% Programme goals for the 5-year education programme in Computer Science
% and Engineering, Chalmers University of Technology, Sweden.

\paragraph{Goal 1:} Based on proven experience as well as insights of
current research and development, the programme graduate should be
able to demonstrate knowledge in:

%\begin{itemize}
% \item 
Hardware / software interaction: explain, model and demonstrate how to
control computer systems using existing and future computer hardware
and other computational manners.

% \item 
Computer programming and the software development processes.

% \item 
Theory of computation and inter-computer interaction: algorithms,
complexity, intractability, logic of a single and many computational
units.

% \item 
Hardware design and development process. \comment{Not development
  process in the sense of clean-room - rather the process from needs,
  specification, high-level design, low-level design, possibly all the
  way to hardware.}
% \end{itemize}

\paragraph{Mål 2:} visa brett kunnande inom mjukvara och hårdvara,
inbegripet kunskaper i matematik och naturvetenskap som anknyter till
datateknikområdet, samt väsentligt fördjupade kunskaper inom vissa
delar av datatekniken.

% {- Kommentarer:
% * Man kan diskutera om bredden är tillräcklig inom följande områden:
% ** Naturvetenskap.
% ** Inom teknikområdet: Distribuerade system. Tillämpningar inom olika
%    datateknikområden.
% * Kopplingen kunde vara större mellan matematikkurserna och
%   datateknikområdena, till exempel vad gäller förmågan att dokumentera
%   matematiska resonemang.
% -}

\paragraph{Mål 3:} Visa förmåga att med helhetssyn kritiskt,
självständigt och kreativt identifiera, formulera och hantera komplexa
frågeställningar samt att delta i forsknings- och utvecklingsarbete
och därigenom bidra till kunskapsutvecklingen.
%
Helhetssyn betyder här: Både datateknikens hårdvaru- och
mjukvaruaspekter såväl som dessas ekonomiska, organisatoriska och
hållbara aspekter. Starka färdigheter inom teknikvetenskapen,
naturvetenskapen och vissa färdigheter inom samhällsvetenskapen.

\paragraph{Mål 4:} Visa förmåga att skapa, analysera, testa och
verifiera dator- och programsystem.

\comment{Några kommentarer:
* Vi diskuterade om "skapa" borde förfinas till t ex designa och
  konstruera, men tyckte det inte var så viktigt.
* Vi diskuterade en läsning av målet som betonar jämförelse mellan
  olika lösningar och om den aspekten kom bort i vår omformulering.
* Vi tyckte att detta mål är lätt att arbeta vidare med och bryta ner
  i delmål för många kurser.
}

\paragraph{Goal 5:} Demonstrate the ability to use analysis and
abstraction methods to plan, design, implement and verify software and
hardware systems that satisfy requirements and are constrained by
available resources.
% Variant
% \paragraph{Goal 5:} Be able to lead, design, implement and verify
% complex projects within specified technological and time-limiting
% constraints, through project planning, modeling, testing, proving and
% verification.

\paragraph{Goal 6:} The students should be able to demonstrate the
ability to 1) critically and systematically integrate and communicate
knowledge and 2) model, simulate, predict, evaluate and interact
with/improve processes from various technical as well as social fields
with limited information.

% {- Suggestion: Look at and use ACM Computer Science Curriculum 2008
% instead of inventing the wheel…
% 
% Patrik: Good idea, but part of the purpose of the meeting was to make
% the Swedish examination rules known to the faculty. 
% -}

\paragraph{Goal 7:} Demonstrate engineering and professional abilities
to model, develop and design computer-related products, software
development processes and computer systems, which take into
consideration human capacity, the society goals and sustainable
developments. In particular, technical and professional skills related
to mathematical, engineering, inter-personal communication, leadership
and ethics.

\paragraph{Mål 8:} visa förmåga till lagarbete och samverkan i grupper
med olika storlekar och sammansättning (innehållandes personer med
olika bakgrund)

\paragraph{Mål 9:} visa förmåga att i såväl nationella som
internationella sammanhang muntligt och skriftligt i dialog med olika
grupper klart redogöra för och diskutera sina slutsatser och den
kunskap och de argument som ligger till grund för dessa.
\begin{itemize}
\item Grupper: Andra ingenjörer i samma ämne, ingenjörer i närliggande
  ämnen, relation till avnämnare.
\item Internationellt: Internationellt campus, internationella
  grupperingar, papers till internationella konferenser.
\item Presentationer, skrivandet. Teknologen ska kunna kommunicera
  till andra vad den uppnått och gjort. 
\end{itemize}

\paragraph{Mål 10:} visa förmåga att göra bedömningar med hänsyn till
relevanta vetenskapliga, samhälleliga och etiska aspekter samt visa
medvetenhet om etiska aspekter på forsknings- och utvecklingsarbete.

\paragraph{Goal 11:} Demonstrate insights of the potential and
limitations of information and communication technology, its role in
society and the responsibility of people for how it is used, with
regards to social, economical and environmental aspects including the
working environment.

\paragraph{Goal 12:} The students should be able to identify their own
needs for further knowledge and to continuously develop their
competence and make use of peer competence.

%%% Local Variables: 
%%% mode: latex
%%% TeX-master: "TKDAT_programbeskrivning_Datateknik_Chalmers"
%%% End: 
