\documentclass[twocolumn]{article}
\usepackage{a4wide}
\usepackage[utf8]{inputenc}
\usepackage[british,swedish]{babel}
\usepackage{tabularx}
\usepackage{hyperref}
\usepackage{graphicx}
\usepackage[table]{xcolor}
\newcommand{\meta}[1]{{\small \emph{#1}}}
\newcommand{\comment}[1]{}
\newcommand{\TODO}[1]{\marginpar{\footnotesize \raggedright #1}}
% \setlength{\parindent}{0mm}
% \setlength{\parskip}{1ex plus 0.7ex minus 0.7 ex}

\pagestyle{plain}
%\pagestyle{headings}
\hypersetup{pdftex, 
  colorlinks=true, 
  linkcolor=blue, 
  citecolor=blue,
  filecolor=blue, 
  urlcolor=blue, 
  pdftitle={Programbeskrivning för Datateknik, 300hp (Chalmers)}, 
  pdfauthor=Patrik Jansson,
  pdfsubject=, 
  pdfkeywords=
}
\title{Programbeskrivning för Datateknik, 300hp (Chalmers)}
\author{Patrik Jansson, programansvarig}
\date{Utkast 2011-12-01}

\begin{document}
\maketitle
\section{Inledning}
%  {\small (1-2 sidor)}

\begin{quote}
  \foreignlanguage{british}{An engineer is a professional practitioner
    of engineering, concerned with applying scientific knowledge,
    mathematics and ingenuity to develop solutions for technical and
    practical problems. \meta{[Wikipedia:Engineer, 2011-09-01]}}
\end{quote}

Termen Datateknik (D) används i detta dokument som motsvarande det
breda begreppet \foreignlanguage{british}{Computer Science and
  Engineering (CSE)}.

\subsection*{Versionshistoria}
{\small
\begin{description}
\item[2011-11-01:] Translated goals to English (PJ) + added D2012 curriculum (prel.)
\item[2011-10-01:] Smärre justeringar (PJ)
\item[2011-09-01:] Omarbetat av \href{http://www.chalmers.se/cse/EN/people/jansson-patrik}{Patrik Jansson} baserat på material från 
\href{http://wiki.portal.chalmers.se/cse/pmwiki.php/PAD/L%C3%A4rarm%C3%B6teVT2011}{lärarmöte 2011-06-01}.
\item[2010-04-11:] utkast av Peter Lundin
\end{description}
}

\subsection{Syfte} 
\comment{Programsyftet beskriver på ett övergripande sätt vad
  programmet vill uppnå}

Civilingenjörsprogrammet i datateknik syftar till att studenten ska
utveckla kunskaper, färdighet och förhållningssätt som gör att de:
\begin{itemize}
\item på ett ingenjörsmässigt sätt kan delta i och leda utveckling av
  generella datorsystem i en internationell omgivning och i samklang
  med samhälle och miljö.
\item kan lösa forskningsmässiga och komplexa datatekniska problem.
\item är väl förberedda för fortsatta studier på forskarnivå inom
  datateknik.
\end{itemize}
Att arbeta på ett ingenjörsmässigt sätt innebär att
%\begin{itemize}
%\item 
baserat på en vetenskapskaplig grund och beprövad erfarenhet,
%\item
tillsammans med andra utvecklare och med användare av systemet,
%\item 
inom givna ekonomiska och tidsmässiga ramar,
%\item 
utveckla en tillräckligt bra lösning på ett problem.
%\end{itemize}

Utbildningen ska även ge goda möjligheter till utveckling av de
personliga egenskaper och attityder som bidrar till god ingenjörsetik.

\subsection{Examen}
\comment{behörighet, examensbenämning, uppläggningen av programbeskrivningen}

Programmet leder till en civilingenjörsexamen i datateknik, samt
normalt också en kandidat\-examen och en masterexamen.

\section{Programme goals}
%  {\small (1-2 sidor)}
\comment{Programmålen utgör de kunskaper, färdigheter och
  förhållningssätt som de studerande förväntas ha utvecklat efter
  genomgången utbildning. Programmålen kan sägas vara en
  konkretisering av syftet och beskriver en mer detaljerad och
  uppföljningsbar uppsättning lärandemål. (Skall vara oberoende av
  implementeringen i termer av kurser.)}

\selectlanguage{british}
The 12 goals of the CSE-programme are specialisations of the degree
ordinance goals (examensordningens mål för civilingenjörsexamen).
% D-programmets 12 mål är specialiseringar av examensordningens mål för
% civilingenjörsexamen.

% En civilingenjör i datateknik ska ha en bred kompetens inom datateknik
% och ska:
% \begin{itemize}
% \item på ett ingenjörsmässigt sätt, kunna delta i och leda utveckling
%   av komplexa datorsystem bestående av hård- eller mjukvara, såväl för
%   generella som för tekniska system.
% \item kunna genomföra utvecklingsarbeten i både nationella och
%   internationella miljöer och i samklang med samhälle och miljö.
% \item visa goda teoretiska kunskaper samt praktiska färdigheter.
% \item kunna lösa forskningsmässiga och komplexa datatekniska problem.
% \end{itemize}

%\cleardoublepage
% Programme goals for the 5-year education programme in Computer Science
% and Engineering, Chalmers University of Technology, Sweden.

\paragraph{Goal 1:} Based on proven experience as well as insights of
current research and development, the programme graduate should be
able to demonstrate knowledge in:

%\begin{itemize}
% \item 
Hardware / software interaction: explain, model and demonstrate how to
control computer systems using existing and future computer hardware
and other computational manners.

% \item 
Computer programming and the software development processes.

% \item 
Theory of computation and inter-computer interaction: algorithms,
complexity, intractability, logic of a single and many computational
units.

% \item 
Hardware design and development process. \comment{Not development
  process in the sense of clean-room - rather the process from needs,
  specification, high-level design, low-level design, possibly all the
  way to hardware.}
% \end{itemize}


% {- General degree statement:
% Goal 1: demonstrate knowledge in the specific technical subject's
% scientific base and proven experience as well as insights of current
% research and development
% 
% Mål 1: visa kunskap om det valda teknikområdets vetenskapliga grund
% och beprövade erfarenhet samt insikt i aktuellt forsknings- och
% utvecklingsarbete
% -}

\paragraph{Goal 2:} demonstrate broad knowledge in software and
hardware, including mathematics and science areas connected to the CSE
area, as well as significant depth of knowledge in certain parts of
the area (the chosen specialisation / MSc programme).
% Swedish version:
% \paragraph{Mål 2:} visa brett kunnande inom mjukvara och hårdvara,
% inbegripet kunskaper i matematik och naturvetenskap som anknyter till
% datateknikområdet, samt väsentligt fördjupade kunskaper inom vissa
% delar av datatekniken.

% {- Kommentarer:
% * Man kan diskutera om bredden är tillräcklig inom följande områden:
% ** Naturvetenskap.
% ** Inom teknikområdet: Distribuerade system. Tillämpningar inom olika
%    datateknikområden.
% * Kopplingen kunde vara större mellan matematikkurserna och
%   datateknikområdena, till exempel vad gäller förmågan att dokumentera
%   matematiska resonemang.
% -}

% {- General degree statement:
% 
% Goal 2: demonstrate both a wide knowledge in the specific technical
% subject, including mathematics and science, as well as significant
% depth of knowledge in certain parts of the subject
% 
% Mål 2: visa såväl brett kunnande inom det valda teknikområdet,
% inbegripet kunskaper i matematik och naturvetenskap, som väsentligt
% fördjupade kunskaper inom vissa delpar av området
% 
% -}

\paragraph{Goal 3:} demonstrate the ability to with a holistic view
identify, formulate and handle complex issues in a critical,
independent and creative manner, and be able to participate in
research and development and thereby contribute to the development of
the field
%
Holistic view here means: Both the hardware- and software-aspects of
CSE as well as their economic, organi\-sational and sustainable aspects.
Strong abilities within engineering science, natural science and some
abilities within social science.

% Swedish version:
% \paragraph{Mål 3:} Visa förmåga att med helhetssyn kritiskt,
% självständigt och kreativt identifiera, formulera och hantera komplexa
% frågeställningar samt att delta i forsknings- och utvecklingsarbete
% och därigenom bidra till kunskapsutvecklingen.
% %
% Helhetssyn betyder här: Både datateknikens hårdvaru- och
% mjukvaruaspekter såväl som dessas ekonomiska, organisatoriska och
% hållbara aspekter. Starka färdigheter inom teknikvetenskapen,
% naturvetenskapen och vissa färdigheter inom samhällsvetenskapen.
% %Oklart för mig vad "färdigheter" betyder inom samhällsvetenskapen
% 
% 
% {- General degree statement:
% Goal 3: demonstrate the ability to with a holistic view identify,
% formulate and handle complex issues in a critical, independent and
% creative manner, and be able to participate in research and
% development and thereby contribute to the development of the field
% 
% Mål 3: visa förmåga att med helhetssyn kritiskt, självständigt och
% kreativt identifiera, formulera och hantera komplexa frågeställningar
% samt att delta i forsknings- och utvecklingsarbete och därigenom bidra
% till kunskapsutvecklingen
% -}

\paragraph{Goal 4:} Demonstrate the ability to create, analyze,
test and verify computer and software systems.
% \paragraph{Mål 4:} Visa förmåga att skapa, analysera, testa och
% verifiera dator- och programsystem.

\comment{Några kommentarer:
* Vi diskuterade om "skapa" borde förfinas till t ex designa och
  konstruera, men tyckte det inte var så viktigt.
* Vi diskuterade en läsning av målet som betonar jämförelse mellan
  olika lösningar och om den aspekten kom bort i vår omformulering.
* Vi tyckte att detta mål är lätt att arbeta vidare med och bryta ner
  i delmål för många kurser.
}

% 
% {- General degree statement:
% 
% Goal 4: demonstrate the ability to create, analyze and critically
% evaluate different technical solutions
% 
% Mål 4: visa förmåga att skapa, analysera och kritiskt utvärdera olika
% tekniska lösningar
% 
% -}
% 

\paragraph{Goal 5:} Demonstrate the ability to use analysis and
abstraction methods to plan, design, implement and verify software and
hardware systems that satisfy requirements and are constrained by
available resources.
% Variant
% \paragraph{Goal 5:} Be able to lead, design, implement and verify
% complex projects within specified technological and time-limiting
% constraints, through project planning, modeling, testing, proving and
% verification.

% 
% {- General degree statement:
% 
% Goal 5: demonstrate the ability to use appropriate methods to plan and
% carry out qualified tasks within given constraints
% 
% Mål 5: visa förmåga att planera och med adekvata metoder genomföra
% kvalificerade uppgifter inom givna ramar
% 
% -}

\paragraph{Goal 6:} The students should be able to
%
1) critically and systematically integrate and communicate knowledge
and 
%
2) model, simulate, predict, evaluate and interact with/improve
processes from various technical as well as social fields with limited
information.

% {- Suggestion: Look at and use ACM Computer Science Curriculum 2008
% instead of inventing the wheel…
% 
% Patrik: Good idea, but part of the purpose of the meeting was to make
% the Swedish examination rules known to the faculty. 
% -}


% {- General degree statement:
% 
% Goal 6: demonstrate the ability to critically and systematically
% integrate knowledge and demonstrate ability to model, simulate,
% predict and evaluate processes with limited information
% 
% Mål 6: visa förmåga att kritiskt och systematiskt integrera kunskap
% samt visa förmåga att modellera, simulera, förutsäga och utvärdera
% skeenden även med begränsad information
% 
% -}

\paragraph{Goal 7:} Demonstrate engineering and professional abilities
to model, develop and design computer-related products, software
development processes and computer systems, which take into
consideration human capacity, the society goals and sustainable
developments. In particular, technical and professional skills related
to mathematical, engineering, inter-personal communication, leadership
and ethics.
%There is nothing "particular" about that last sentence - it is very broad

% {- General degree statement:
% 
% Goal 7: demonstrate the ability to develop and design products,
% processes and systems which are adapted to the needs and capacities of
% their users and which meet the societal requirements of sustainable
% development in economical, social and ecological terms
% 
% Mål 7: visa förmåga att utveckla och utforma produkter, processer och
% system med hänsyn till människors förutsättningar och behov och
% samhällets mål för ekonomiskt, socialt och ekologiskt hållbar
% utveckling
% -}

\paragraph{Goal 8:} Demonstrate the ability to work in teams and to
collaborate in groups with different sizes and constitutions
(consisting of individuals with different backgrounds).

% Svensk version:
% \paragraph{Mål 8:} visa förmåga till lagarbete och samverkan i grupper
% med olika storlekar och sammansättning (innehållandes personer med
% olika bakgrund)

% {- General degree statement:
% 
% Goal 8: demonstrate the ability to work in teams and to collaborate in
% groups with different constitutions
% 
% Mål 8: visa förmåga till lagarbete och samverkan i grupper med olika
% sammansättning
% -}

\paragraph{Goal 9:} Demonstrate the ability to discuss and give a
clear account of his/her conclusions and the knowledge and arguments
that support these both orally and in writing addressing different
audiences in both national and international contexts.
\begin{itemize}
\item Groups: Other engineers in the same subject, engineers in nearby
  subjects, to some degree stakeholders / customers.
\item International: International campus, international groups,
  papers for international contexts.
\end{itemize}
% \paragraph{Mål 9:} visa förmåga att i såväl nationella som
% internationella sammanhang muntligt och skriftligt i dialog med olika
% grupper klart redogöra för och diskutera sina slutsatser och den
% kunskap och de argument som ligger till grund för dessa.
% \begin{itemize}
% \item Grupper: Andra ingenjörer i samma ämne, ingenjörer i närliggande
%   ämnen, i viss mån avnämnare / kunder.
% \item Internationellt: Internationellt campus, internationella
%   grupperingar, artiklar till internationella konferenser.
% \end{itemize}
% 
% {- General degree statement:
%
% Goal 9: demonstrate the ability to discuss and give a clear account of
% his/her conclusions and the knowledge and arguments that support these
% both orally and in writing addressing different audiences in both
% national and international contexts
% 
% Mål 9: visa förmåga att i såväl nationella som internationella
% sammanhang muntligt och skriftligt i dialog med olika grupper klart
% redogöra för och diskutera sina slutsatser och den kunskap och de
% argument som ligger till grund för dessa
% 
% -}

\paragraph{Goal 10:} Demonstrate the ability to make assessments
taking into account relevant scientific, societal and ethical aspects
and show awareness of ethical aspects on research and development.
% \paragraph{Mål 10:} visa förmåga att göra bedömningar med hänsyn till
% relevanta vetenskapliga, samhälleliga och etiska aspekter samt visa
% medvetenhet om etiska aspekter på forsknings- och utvecklingsarbete.

% {- General degree statement:
% 
% Goal 10: demonstrate the ability to make assessments taking into
% account relevant scientific, societal and ethical aspects and show
% awareness of ethical aspects on research and development
% 
% Mål 10: visa förmåga att göra bedömningar med hänsyn till relevanta
% vetenskapliga, samhälleliga och etiska aspekter samt visa medvetenhet
% om etiska aspekter på forsknings- och utvecklingsarbete
% -}

\paragraph{Goal 11:} Demonstrate insights of the potential and
limitations of information and communication technology, its role in
society and the responsibility of people for how it is used, with
regards to social, economical and environmental aspects including the
working environment.

% {- General degree statement:
% 
% Goal 11: demonstrate insights of the potential and limitations of
% technology, its role in society and the responsibility of people for
% how it is used, with regards to social, economical and environmental
% aspects including the working environment.
% 
% Mål 11: visa insikt i teknikens möjligheter och begränsningar, dess
% roll i samhället och människors ansvar för hur den används, inbegripet
% sociala och ekonomiska aspekter samt miljö- och arbetsmiljöaspekter
% 
% -}
\paragraph{Goal 12:} The students should be able to identify their own
needs for further knowledge and to continuously develop their
competence and make use of peer competence.

% {- General degree statement:
% 
% Goal 12: demonstrate the ability to identify his/her needs for further
% knowledge and to continuously develop his/her competence
% 
% Mål 12: visa förmåga att identifiera sitt behov av ytterligare kunskap
% och att fortlöpande utveckla sin kompetens
% -}
% 


%%% Local Variables: 
%%% mode: latex
%%% TeX-master: "TKDAT_programbeskrivning_Datateknik_Chalmers"
%%% End: 


\subsection{Goals of the associated MSc programmes}
\subsubsection{Computer Science --- Algorithms, Languages and Logic}
% Goals for "Computer Science --- Algorithms, Languages and Logic"
\meta{Version 2011-12-01 by Peter Damaschke.}

% This is in a separate file to make it easy to include from both the
% MPALG and TKDAT programme descriptions.
Computer systems are becoming increasingly powerful and intelligent,
and they rely on increasingly sophisticated techniques. To master the
complexity of these systems it is essential to understand the core
areas of computer science. While computer systems and their
applications develop rapidly, the underlying foundations of computing
are mature mathematical theories.

This programme offers a comprehensive foundation in the science of
programming and computing, thus providing lasting knowledge in the
field. It gives the student a strong basis for developing the computer
applications of today and tomorrow and for conducting innovative
research and promoting development.

After completion of their studies, the students will be able to

\paragraph{CSALL1} apply solid knowledge in the mathematical and logical
  foundations of computing, to computational problems appearing, e.g.,
  in industry and the public sector, thereby taking into account
  aspects like tractability, complexity, correctness, and security,
\paragraph{CSALL2} create models of real-world scenarios that are both meaningful
  and amenable to analysis, thereby choosing the right level of
  abstraction,
\paragraph{CSALL3} develop correct and efficient computer programmes and systems
  that satisfy given requirements under given constraints,
\paragraph{CSALL4} analyse and test systems, evaluate, predict, prove and verify
  their essential properties,
\paragraph{CSALL5} identify and handle complex computational problems,
\paragraph{CSALL6} communicate their ideas and results to both specialists and
  nonspecialists, give structured and scholarly presentations in oral
  and written form,
\paragraph{CSALL7} find scientific information and integrate knowledge, identify
  their own needs for gathering further knowledge and do the necessary
  self-study,
\paragraph{CSALL8} work in project groups and international environments, take a
  leading role,
\paragraph{CSALL9} critically judge systems, results, and the use of information
  technology also from a social point of view, and be aware that
  results of computations crucially depend on model assumptions.



\subsubsection{Computer Systems and Networks}

\hyphenation{Adv-Dist-Sys Par-Dist-RTS}

\meta{Version 2011-12-07 by Elad Schiller.}

All of the courses in the MPCSN programme have being running for
several years and they all provide the student with important
engineering skills. Each course in the programme includes several home
assignments, labs, programming projects and exercises. In addition,
key professional skills are examined in the final written exams that
each course has.

A partial list the professional skills that the CSN programme provides
are listed below.

\paragraph{CSN1} Demonstrate knowledge in computer systems and
networks with emphasis on algorithm design, and computer programming
and with proven experience as well as insights of current research and
development.

Courses: FaultTol, CompNet, AdvCSN, DistSys, RealTime.

\paragraph{CSN2} Demonstrate the ability to develop and design
computer algorithms, programs, network protocols, services and
innovation in these areas, which are adapted to the needs and
capacities of their users, and which meet the societal requirements of
sustainable development in economical, social and ecological terms.

Courses: CompNet, AdvDistSys, ParDistRTS, CompSec, NetSec, AdvCSN,
MScAoA.

\paragraph{CSN3} Demonstrate both a wide knowledge in the specific
technical subject, including mathematics and related sciences, as well
as significant depth of knowledge in designing algorithms and
programming building computer systems and networks.

Courses: AdvDistSys, ParDistRTS, NetSec, Route, AdvCSN, MScAoA.

\paragraph{CSN4} Demonstrate the ability to work in teams and to
collaborate in groups with different constitutions and backgrounds in
algorithms, programming and cyber-physical systems.

Courses: ParDistRTS, MScAoA, MScThesis.

\paragraph{CSN5} Demonstrate the ability to discuss and give a clear
account of his/her conclusions and the knowledge and arguments that
support these both orally and in writing addressing different
audiences in both national and international contexts.

Courses: FaultTol, AdvCSN, MScAoA, MScThesis.

\paragraph{CSN6} Demonstrate the ability to create, analyse and
critically evaluate different algorithmic designs and their
implementations in computer systems and networks.

Courses: FaultTol, CompNet, AdvDistSys, ParDistRTS, CompSec, NetSec,
AdvCSN, MScAoA, MScThesis.

\paragraph{CSN7} Demonstrate the ability to make assessments of
computer programs, and network protocols while taking into account
relevant analytical, societal and ethical aspects and show awareness
of ethical aspects on research and development, and to demonstrate
insights of the potential and limitations of computer and networks
technologies.

Courses: CompSec, AdvCSN.

\paragraph{CSN8} Demonstrate the ability, when given system settings,
to design efficient algorithms and program them in a variety of such
systems settings.

Courses: CompNet, AdvDistSys, RealTime, ParDistRTS, AdvCSN.

\paragraph{CSN9} Demonstrate insights of the potential and limitations
of computer and networks technologies, their role in society and the
designer responsibility for their use, with regards to social,
economical and environmental aspects including the working
environment.

Courses: FaultTol, CompNet, DistSys, AdvDistSys, CompSec, NetSec,
Route.

\paragraph{CSN10} Demonstrate the ability to identify his/her needs
for further knowledge and to continuously develop his/her competence,

Courses: FaultTol, CompNet, AdvDistSys, ParDistRTS, CompSec, NetSec,
AdvCSN, MScAoA, MScThesis.

\paragraph{CSN11} Plan and with adequate methods carry out qualified
tasks within given constraints or within a given framework. Within the
framework of computer systems, the considered methods are from the
areas of algorithm design and computer programming. Resources
efficiency is the main design and implementation constraint along with
usability of computer systems.

Courses: FaultTol, CompNet, DistSys, AdvDistSys, RealTime,
ParDistRTS, CompSec, NetSec, Route, AdvCSN, MScAoA, MScThesis.

\paragraph{Course abbreviations:}
\begin{description}
\item[AdvCSN    ] Advanced topics in computer systems and networks  (DAT145)
\item[AdvDistSys] Advanced Distributed systems                      (TDA297) 
\item[CompNet   ] Computer networks                                 (EDA387) 
\item[CompSec   ] Computer security                                 (EDA263) 
\item[DistSys   ] Distributed systems                               (TDA596) 
\item[FaultTol  ] Fault tolerant computer systems                   (DAT270) 
\item[MScAoA    ] Masterclass: Area of Advance                      (DATnnn) 
\item[NetSec    ] Network security                                  (EDA491) 
\item[ParDistRTS] Parallel and distributed real-time systems        (EDA421) 
\item[RealTime  ] Real time systems                                 (EDA222) 
\item[Route     ] Routing technology                                (LEU420) 
\end{description}

% \begin{tabular}{lcp{.6\columnwidth}}
%    AdvCSN     &=& DAT145 Advanced topics in computer systems and networks,
% \\ AdvDistSys &=& TDA297 Advanced Distributed systems,
% \\ CompNet    &=& EDA387 Computer networks,
% \\ CompSec    &=& EDA263 Computer security,
% \\ FaultTol   &=& DAT270 Dependable computer systems,
% \\ DistSys    &=& TDA596 Distributed systems,
% \\ MScAoA     &=& DATnnn Masterclass: Area of Advance,
% \\ NetSec     &=& EDA491 Network security,
% \\ ParDistRTS &=& EDA421 Parallel and distributed real-time systems,
% \\ RealTime   &=& EDA222 Real time systems,
% \\ Route      &=& LEU420 Routing technology.
% \end{tabular}


\selectlanguage{swedish}
\section{Programutformning}
%  {\small (5-8 sidor)}
\newcommand{\courselink}[2]{\href{https://www.student.chalmers.se/sp/course?course_id=#1}{#2}}

\newcommand{\introFP }{\courselink{16717}{Introduktion till funktionell programmering}}
\newcommand{\DiskMat }{\courselink{16714}{Inledande diskret matematik}}
\newcommand{\DigoDat }{\courselink{16012}{Grundläggande datorteknik (ny)}}
\newcommand{\LinAlg  }{\courselink{16376}{Linjär algebra}}
\newcommand{\MOP     }{\courselink{15446}{Programmering av inbyggda system (ny)}}
\newcommand{\MatAn   }{\courselink{15413}{Matematisk analys}}
\newcommand{\OOP     }{\courselink{16190}{Objektorienterad programmering (flyttad)}}
\newcommand{\ElKrets }{\courselink{15353}{Elektriska kretsar och fält}}

\newcommand{\DatProj }{\courselink{TODO} {Datatekniskt projekt (ny)}}
\newcommand{\MaStDiMa}{\courselink{16511}{Matematisk statistik och diskret matematik}}
\newcommand{\DStr    }{\courselink{16480}{Datastrukturer}}
\newcommand{\FyIng   }{\courselink{16409}{Fysik för ingenjörer}}
\newcommand{\DigSynt }{\courselink{16419}{Digitalteknik-syntes}}
\newcommand{\ParProg }{\courselink{15402}{Parallell programmering}}
\newcommand{\ProgPara}{\courselink{16828}{Programmerings\-paradigmer}}
\newcommand{\DaSyTe  }{\courselink{15460}{Datorsystemteknik}}
\newcommand{\Datakom }{\courselink{16479}{Datakommunikation}}

\newcommand{\Barkraft}{\courselink{16340}{? Ny Bärkraftig resursanvändning}}
\newcommand{\TraSiSy }{\courselink{16710}{? Ny (Transformer, signaler och system)}}
\newcommand{\Regler  }{\courselink{15591}{? Ny Reglerteknik}}
\newcommand{\Kandarb }{\courselink{16697}{Kandidatarbete}}

\begin{figure*}
  \centering
\newcolumntype{C}{>{\centering\arraybackslash}X}


\textbf{Årskurs 1}

\begin{tabularx}{\textwidth}{|*{4}{C|}}
  \hline
  \introFP{} &  \DigoDat{} &  \OOP{}   &  \MOP{}      \\\hline
  \DiskMat{} &  \LinAlg{}  &  \MatAn{} &  \ElKrets{}  \\\hline
\end{tabularx}

\textbf{Årskurs 2}

\begin{tabularx}{\textwidth}{|*{4}{C|}}
  \hline
  \DatProj{}  &  \DStr{}   &  \DigSynt &  \DaSyTe{}   \\\hline
  \MaStDiMa{} &  \FyIng{}  &  \ParProg{} \emph{eller} \ProgPara{} & \Datakom{}
  \\\hline
\end{tabularx}

\textbf{Årskurs 3}

\begin{tabularx}{\textwidth}{|*{4}{C|}}
  \hline
  \Regler{}   &
  \Barkraft{} &
  \Kandarb{}  &
  \Kandarb{} forts. 
\\\hline
  Valfri kurs &
  Valfri kurs &
  Valfri kurs &
  Valfri kurs
\\\hline
\end{tabularx}
  
  \caption{Årskurs 1--3, D2012 (preliminärt).}
  \label{fig:year123}
\end{figure*}
\subsection{Programidé}
\comment{Programidén redogör för hur utbildningen är upplagd för att
  uppnå de formulerade målen. Den är typiskt i form av en punktlista
  med de viktigaste principerna bakom programmets utformning, eller
  valda strategier för att uppnå målen.}

Programmet skall genom sin utformning ge alla studenter på programmet
en möjlighet och förmåga att utvecklas kunskapsmässigt inom
framförallt ett brett fält av datateknik samt individuellt som person
inom områden som ger stöd i den kommande yrkesrollen som ledande
teknikutvecklare.
%Notera att datalogi ingår i det vi menar med datateknik i detta dokument.

Programmet ska ge en god grund för att i den kommande
yrkesverksamheten, på olika nivåer, aktivt kunna medverka i och leda
utveckling av hård- och framförallt mjukvara för både inbyggda och
generella datorsystem. Detta kräver teoretiska grundkunskaper inom
både datateknik och matematik.

Programmet har en struktur enligt Bologna-modellen med en första cykel
om tre år på grundläggande nivå (som leder till kandidatexamen) samt
en efterföljande cykel om 2 år med kurser på avancerad nivå (inom
masterprogram). Den senare cykeln av utbildningen genomförs helt på
engelska tillsammans med internationella studenter och syftar till att
ge fördjupade kunskaper inom en begränsad del av ämnesområdet
datateknik för att skapa möjligheten att kunna leda och medverka i
utveckling av mer komplexa datorsystem.

Genom att i de första två åren nästan enbart erbjuda obligatoriska
kurser följt av ett tredje år med en relativt stor valbarhet ges en
tillräcklig gemensam ämnesplattform som karakteriserar en
civilingenjör i datateknik samt en valbarhet för att testa olika
områden inför den kommande specialiseringen i de avslutande två åren.
Det tredje året avslutas med ett kandidatarbete inom vilket man i en
projektgrupp genomför en projektuppgift vanligen i form av ett
konstruktionsuppdrag. Kandidatarbetet skall ge tillfälle att kombinera
kunskaperna från olika tidigare kurser samt ge träning i olika faser
av ett utvecklingsprojekt.

\subsection{Programplan}

\comment{Programplanen beskriver exakt vilka kurser som ska läsas som
  obligatoriska eller är valbara för varje årskurs av utbildningen som
  startat ett visst läsår.}

Programmet är indelat i kurser omfattande 7.5hp av vilka man läser
två per läsperiod.
\paragraph{Årskurs 1 till 3} De tre första årens kurser är placerade
enligt figur \ref{fig:year123} där varje kolumn motsvarar en läsperiod.

\paragraph{Årskurs 4 och 5}

Under de avslutande två åren följer studenterna ett av nedan angivna
masterprogram.  Programmen skiljer sig åt avseende förkunskaper och
platsgaranti.  Masterprogrammen består normalt av 60 hp (delvis)
obligatoriska kurser som tillsammans med det avslutande
masterexamensarbetet om 30 hp säkerställer fördjupningen inom den
valda specialiseringen. (Det finns även möjlighet till ett
examensarbete om 60hp under särskilda omständigheter.)

För det övriga innehållet i masterprogrammen erbjuds en stor valfrihet
avseende kurser och kursinnehåll i syfte att erbjuda studenten en
möjlighet till en personligt styrd breddning eller fördjupning av
utbildningen.

För att bli behörig behöver man förutom de allmänna kraven dessutom
läsa vissa angivna kurser under det tredje läsåret. För information om
förkunskapskrav och övrig information hänvisas till respektive
masterprograms utbildningsplan som man hittar via Chalmers
hemsida. Till masterprogrammen antas även studenter från andra svenska
eller internationella högskolor och universitet.  Undervisning och
examination på masterprogrammens kurser genomförs normalt på engelska.

De masterprogram som erbjuds inom datateknik är de nedan
angivna. D-studenter som uppfyller antagningskraven har garantiplats
på något av de två program märkta med (Garanti).

\begin{itemize}
\item \foreignlanguage{british}{Computer Science --- Algorithms,
    Languages and Logic} (Garanti)
\item \foreignlanguage{british}{Computer Systems and Networks} (Garanti)
\item \foreignlanguage{british}{Software Engineering}
\item \foreignlanguage{british}{Interaction Design and Technologies}
\item \foreignlanguage{british}{Embedded Electronic System Design}
\item Lärande och ledarskap (som också leder till gymnasielärarexamen)
\item \foreignlanguage{british}{Systems, Control and Mechatronics}
\item \foreignlanguage{british}{Engineering Mathematics and Computational Science}
\end{itemize}

\subsection{Lärsekvenser / program\-designmatris}
\newcommand{\rot}[1]{\rotatebox{90}{#1}}
\begin{figure*}
\centering
\rowcolors{1}{white}{lightgray}
\begin{tabular}{rlllllllllllllllllll}
   \rot{$\leftarrow$Mål / $\downarrow$Kursnamn}
  &\rot{\introFP{}} 
  &\rot{\DiskMat{}}
  &\rot{\DigoDat{}}
  &\rot{\LinAlg{}}
  &\rot{\MOP{}}
  &\rot{\MatAn{}}
  &\rot{\ElKrets{}}
  &\rot{\OOP{}}
  &\rot{\DatProj{}}
  &\rot{\MaStDiMa{}}
  &\rot{\DStr{}}
  &\rot{\FyIng{}}
  &\rot{\DigSynt}
  &\rot{\DaSyTe{}}
  &\rot{\Datakom{}}
  &\rot{\Barkraft{}}
%  &\rot{\TraSiSy{}}
  &\rot{\Regler{}}
  &\rot{\Kandarb{}}
  &\rot{Masterprogram}
%   FpDmDdLaMpMaElOpDpMsDsFiDiDtDkBäReKaMa% Ts
\\ 1&X&X&X& &X& &X&X&X&X&X&X&X&X&X&X& &X&X% &X
\\ 2&X&X&X&X&X&X&X&X& &X&X&X&X&X&X&X&X&X&X% &X
\\ 3&X&X&X&X&X&X&X&X& &X&X&X&X&X&X&X&X&X&X% &X
\\ 4&X& &X& &X& & &X& & &X& &X&X&X& & &X&X% & 
\\ 5& & & & & & & & &X& & & & & & & & &X&X% & 
\\ 6& & & & & & &X& &X&X& &X&X& & & & &X&X% &X
\\ 7& & &X&X& & & & &X& &x& & & & & &X&x&X% & 
\\ 8&x& & & & & & & &X& & & & & & & & &X&X% & 
\\ 9& & & & & & & & &X& &x& & & & & & &X&X% &X
\\10& & & & & & & & & & & & & & & &X& &x&x% & 
\\11& & & & & & & & & & & &x&x& & & & & & % & 
\\12& & & & & & & & & & & &X& & & & & & &X% & 
  \end{tabular}
  \caption{Programdesignmatris D2012}
  \label{fig:programdesignmatris}
\end{figure*}

\comment{Programdesignmatrisen kopplar samman programmets mål med de
  ingående kurserna så att det framgår i vilken/vilka kurs(er) som
  vissa kunskaper eller färdigheter tänks utvecklas och examineras. I
  programdesignmatrisen synliggörs också planerade lärsekvenser för
  kunskaper, färdigheter och förhållningssätt som utvecklas som
  integrerade inslag i en serie av kurser.
\begin{itemize}
\item programmets mål är raderna, programmets kurser kolumnerna
\item I = Introducera, U = Undervisa, A = Använda anges i rutorna
\item alternativt kan man direkt göra en mer effektiv beskrivning av
  den glesa matrisen genom att bara lista ``lärsekvenser''
\end{itemize}
}

I figur \ref{fig:programdesignmatris} finns en preliminär
programdesignmatris där ``X'' står för ``I'', ``U'', eller ``A'' och
där lilla ``x'' är en svagare koppling är stora ``X''.

\appendix

% \href{http://wiki.portal.chalmers.se/cse/pmwiki.php/PAD/Programbeskrivningsprocess}{Wiki-sida för programbeskrivningsprocessen}
% 
\section{TODO}
\begin{itemize}
\item fyll på med lärsekvenser
\item jämför studentportalen förkunskapskrav och lärmål på alla ingående kurser
\end{itemize}
\section{Vision om D-program\-mets framtid}

\meta{Patrik: Detta är delar av den text jag skrev i ansökan för att
  bli PA för Datateknik november 2010.}
%
% Vi lever i en oerhört spännande tid när datatekniken infiltrerar
% snart sagt varje del av samhället.
% %
% Vi är alla användare av datatekniska produkter och tjänster och det
% är svårt att tänka sig vårt moderna samhälle utan mobiltelefoner,
% Internet och ABS-bromsar.
% %
% För användare räcker det att tekniken fungerar (och ser bra ut) men i
% bakgrunden måste någon planera, utveckla, implementera och underhålla
% (de ofta programvarubaserade) systemen.
% %
% Det är där som huvudfokus bör vara för D-programmet nu och i framtiden.
% %
% Samhället (inklusive industrin) behöver dataingenjörer som har koll på
% god kvalitet, hög produktivitet och låga kostnader.
% %
%
% Men en högre utbildning är så mycket mer än så---personlig utveckling,
% nyfikenhet och innovation är också viktiga komponenter.
% %
% Många kommer inte att jobba just med det de lärt sig på Chalmers men
% en gedigen bakgrund i teknik och problemlösning kan räcka långt i
% nästan alla områden.
% %
% Många går vidare till forskning och utveckling och för dem gäller det
% att tidigt hitta utmaningar och ingångar som leder dem vidare.
%
% Slutligen finns också övergripande utmaningar för mänskligheten i stort
% som delproblem inom den globala omställningen till en hållbar
% utveckling.
% %
% Här känner jag att det finns mycket att göra med datateknik som
% verktyg och därmed bör det finnas goda chanser att utveckla
% utbildningen så att detta framgår redan tidigt.
% %

Några till punkter --- jag vill
\begin{itemize}
\item utnyttja studenternas egna erfarenheter och intressen genom att
  erbjuda rejäla möjligheter till valfrihet redan inom de första tre
  åren.
\item se ett fortsatt nära samarbete med IT och med datavetenskap (på
  GU), men även med E, Z, samt SE\&M (på GU).
\item arbeta i nära samråd med berörda institutioner och lärare.
\item främja datorstödd modellering och högnivåspråk inom alla ämnen:
  datavetenskap, datorteknik, mate\-matik, teknikbredd, \ldots
\item utveckla kandidatdelen av programmet baserat på vad
  masterutbildningen behöver (enl. lärare och studenter).
\item utveckla masterprogrammen baserat på vad framtida arbetsgivare
  (industri, akademi, samhälle) behöver och studenter vill.
\item öka rekryteringen av studenter inom EU --- särskilt från
  östeuropa.
\end{itemize}

\end{document}
