\documentclass[twocolumn]{article} 
%\documentclass[11pt]{article} 
%\renewcommand{\baselinestretch}{0.96}
\usepackage{a4wide}
\newcommand{\meta}[1]{{\small \emph{#1}}}
\newcommand{\comment}[1]{}
\newcommand{\TODO}[1]{\marginpar{\footnotesize \raggedright #1}}
% \setlength{\parindent}{0mm}
% \setlength{\parskip}{1ex plus 0.7ex minus 0.7 ex}

\begin{document}
\title{Computer Science---Algorithms, Languages and Logic\\
M.Sc., 120 cr, 2 years}
\author{Peter Damaschke}
\date{2011-12-01}
\maketitle

\section{Programme aim}

% Goals for "Computer Science --- Algorithms, Languages and Logic"
\meta{Version 2011-12-01 by Peter Damaschke.}

% This is in a separate file to make it easy to include from both the
% MPALG and TKDAT programme descriptions.
Computer systems are becoming increasingly powerful and intelligent,
and they rely on increasingly sophisticated techniques. To master the
complexity of these systems it is essential to understand the core
areas of computer science. While computer systems and their
applications develop rapidly, the underlying foundations of computing
are mature mathematical theories.

This programme offers a comprehensive foundation in the science of
programming and computing, thus providing lasting knowledge in the
field. It gives the student a strong basis for developing the computer
applications of today and tomorrow and for conducting innovative
research and promoting development.

After completion of their studies, the students will be able to

\paragraph{CSALL1} apply solid knowledge in the mathematical and logical
  foundations of computing, to computational problems appearing, e.g.,
  in industry and the public sector, thereby taking into account
  aspects like tractability, complexity, correctness, and security,
\paragraph{CSALL2} create models of real-world scenarios that are both meaningful
  and amenable to analysis, thereby choosing the right level of
  abstraction,
\paragraph{CSALL3} develop correct and efficient computer programmes and systems
  that satisfy given requirements under given constraints,
\paragraph{CSALL4} analyse and test systems, evaluate, predict, prove and verify
  their essential properties,
\paragraph{CSALL5} identify and handle complex computational problems,
\paragraph{CSALL6} communicate their ideas and results to both specialists and
  nonspecialists, give structured and scholarly presentations in oral
  and written form,
\paragraph{CSALL7} find scientific information and integrate knowledge, identify
  their own needs for gathering further knowledge and do the necessary
  self-study,
\paragraph{CSALL8} work in project groups and international environments, take a
  leading role,
\paragraph{CSALL9} critically judge systems, results, and the use of information
  technology also from a social point of view, and be aware that
  results of computations crucially depend on model assumptions.



\section{Who should apply}

The programme is intended for students who wish to study the core
areas of computer science on an advanced level in order to prepare
themselves for research and development particularly in the software
industry. It also provides an ideal basis for academic research in
computer science.
 
Most students will have a BSc in computer science. However, the
programme can also serve as a conversion course for students with BSc
in related subjects, such as mathematics, physics or engineering
sciences, provided they have basic knowledge of mathematics and
programming, and have completed an introductory computer science
course such as data structures or algorithms.

\section{Why apply}

Students will obtain a strong computer science background and thus
gain access to a wide range of opportunities in the information
technology industry.  Students acquire generic skills and lasting
subject knowledge and are in a good position to understand and
contribute to technological advances. Products and whole companies are
based on advanced algorithmic techniques, and these industries employ
skilled computer scientists. The programme also provides the students
with an excellent background for future PhD studies in computing,
which can lead to a career as an academic researcher or computing
teacher.

\section{Research connections}

Chalmers pursues internationally recognised research,
also in cooperation with industry, in all core areas of the
programme. Chalmers is well-known for its research in functional
programming and played a major role in the design and development of
the standard lazy functional language Haskell. Powerful software tools
for testing and combinatorial problem solving have widespread
use. Methods from programming language theory are applied to problems
in security. Pioneering work is conducted in type theory and computer
assisted theorem proving. Researchers in programming logic also
collaborate with linguists in the field of natural language
processing. Another important area is the design and analysis of
algorithms and their applications in bioinformatics and networks in a
broad sense, and in large-scale optimisation.  Teachers are active
researchers in these fields, which gives the programme scientific
quality and depth.

\section{Specific eligibility}
\subsection{Undergraduate profile}

Major in Computer Science, Computer Engineering, Mathematics,
Engineering Physics or Information Engineering.
 
\subsection{Prerequisites}

Mathematics (including Discrete Mathematics, Linear Algebra, and
Analysis), Statistics, Programming, basic knowledge of Data
Structures.

\section{Programme content}
 
The core of the programme covers four main branches of computer science:
\begin{itemize}
\item {\em Algorithms}, including artificial intelligence, machine
  learning and optimisation,
\item {\em Logic}, including applications in hardware and software
  verification,
\item {\em Programming languages}, with underlying principles,
  implementation techniques and advanced programming techniques,
\item {\em Computer security}, including cryptography and
  language-based security.
\end{itemize}

Students have to take the following core courses:
\begin{itemize}
\item Algorithms
\item Logic in Computer Science
\item Programming Language Technology
\end{itemize}

The optional segment of the programme offers the student a broad range
of courses in other areas of computer science, bioinformatics,
software engineering, mathematics, and other related fields.
 
At least 5 of the following profile courses, divided in 4 tracks, must
be taken:

\begin{itemize}
\item {\bf Algorithms:} Advanced Algorithms, Discrete Optimization,
  Algorithms for Machine Learning and Inference, Artificial
  Intelligence
\item {\bf Logic and Verification:} Models of Computation, Types for
  Programs and Proofs, Software Engineering using Formal Methods,
  Hardware Description and Verification\footnote{Under discussion, may
    be cancelled.}
\item {\bf Programming Languages:} Compiler Construction, Advanced
  Functional Programming, Frontiers of Programming Language
  Technology, Parallel Functional Programming
\item {\bf Security:} (no core course) Cryptography, Programming
  Language Based Security
\end{itemize}

Students may specialise on a track or aim at a broader education and
mix courses from several tracks. The curriculum is filled with 4
elective courses that can be freely chosen and may also come from
neighbouring programmes. It is also possible to take an elective project
course and PhD level courses.

The programme is completed with a 30 cr master's thesis. The
thesis proposal has to be approved by the programme director.

%\newpage
\section*{Appendix}

The programme has the ambition to offer the students a variety of
courses that are tied to the research strengths at the Department of
Computer Science and Engineering. It provides freedom of choice, thus
enabling individual study plans, but at the same the tracks give
guidance for meaningful paths to go.  The roles of most of the courses
are detailed below.

\bigskip
%
The {\bf compulsory core courses} represent the three cornerstones of the 
programme: Algorithms, Logic in Computer Science, Programming Language 
Technology.

\bigskip
%
Courses in the {\bf Algorithms track} rely on the basic Algorithms course (or
knowledge gained in equivalent courses) and can be studied in any order.

{\bf Advanced Algorithms} is a generic algorithms course that covers
the main techniques of design and analysis of algorithms, but in
selected fields it goes more into depth than the basic course. In
particular it provides solution methods for computationally hard
problems. It adopts a rigorous analytical approach and emphasises
correctness, efficiency, and accuracy.

{\bf Discrete Optimization}, in contrast, focuses on a special but
very important class of problems, integer linear programs (ILP), and
provides methods for modelling (that is, completely specifying)
real-world optimisation problems as ILP, as well as methods for
solving them.

{\bf Algorithms for Machine Learning and Inference} is another
specialised course. It shows how algorithms can extract information
and automatically draw conclusions from data sets, on solid scientific
grounds.

{\bf Artificial Intelligence} (AI) has similar contents but a unique
shape, consisting of a block of introductory lectures followed by
projects to be carried out by small groups of students. Logic appears
as a formalism to specify and solve problems in AI.

{\bf Structural Bioinformatics} is not among the profile courses but
can be taken as a deepening course in Algorithms: Students with an
interest in computational molecular biology learn about problems in
this application domain that are accessible to the algorithmic methods
seen in the other courses.

\bigskip
%
The {\bf Logic and Verification} track builds upon the compulsory
Logic in Computer Science course. There is also some interaction with
Programming Language Technology, thus students will be helped by
having done this compulsory course before. However the profile courses
can be read in any order.

{\bf Models of Computation} is a course about formalisms that make the
intuitive notion of computability explicit.

{\bf Types for Programs and Proofs} gives broad knowledge of type
systems for programming languages, including interactive programming
and proof systems using dependent types.

{\bf Software Engineering using Formal Methods} shows how to formally
express and verify safety properties of programmes. This way it
applies logic to verification tools.

\bigskip
%
The {\bf Programming Languages} track comprises courses on several
aspects of programming languages, which is the main tool to make
computers actually work.

{\bf Compiler Construction} addresses the syntactic structure of
programming languages, including type systems, and translation of
source code into executable code. It can be noticed that problems in
run-time organisation, code optimisation, and register allocation
build a bridge to the field of discrete optimisation.

{\bf Advanced Functional Programming} explores the powerful mechanisms
that functional programming offers to solve real problems and to
structure large programs. A focus is on library design, embedded
languages, and types.  Specification, program properties, and
reasoning about correctness are common themes in this master's
programme.

{\bf Parallel Functional Programming} has been introduced as a course
that contributes to the increasingly important area of parallelism
(e.g., multi-core processors).

\bigskip
%
The {\bf Security} track is not directly built upon a core course but
has obvious relationships to algorithms and problem complexity as well
as programming languages and verification of properties:

{\bf Cryptography} presents, among other topics, public key primitives
based on the computational hardness of mathematical problems, hash
functions as parts of cryptographic protocols, and security
guarantees.

{\bf Programming Language Based Security} is based on the idea that
many attacks work on the application level, and these are typically
specified in programming languages. A direct benefit of language-based
security is the ability to naturally express security policies using
the well-developed area of programming languages.

\bigskip
%
The {\bf Project in Computer Science} is not a course in the usual
sense, nor does it belong to a track. It gives students the
opportunity to do supervised work on a freely chosen subject which,
however, must fit in the programme and be nontrivial.

\end{document}
