\meta{Version 2011-02-23 by Elad Schiller (copied from the
  \href{https://www.student.chalmers.se/sp/program?program_id=814}{student
    portal}).}

One of the programme's teaching philosophies is that a solid grasp of
computer systems and networks can be developed by studying the
different methodologies used to construct computer systems and
networks such as system engineering, computer architecture,
programming and distributed computing. Consequently, course\-work
requires students to solve problems from the area of system and
network design that involve the above mentioned aspects and techniques.

After the completion of the programme, the student will be able to:
\begin{itemize}
\item Design a system based on new and existing components (System
  Engineering),
\item Discuss with a professional the meaning of key concepts in
  low-level hardware/software interaction (Computer Architecture),
\item Develop systems and applications (Programming),
\item Analyse the performance and limitations of the designed system
  (Distributed Computing).
\end{itemize}

%Is there really a "holistic philosophy" in this programme? What does that mean?
The programme's holistic philosophy equips its graduates with a wide
range of industry-related engineering skills. Rather than
concentrating on a single aspect of computer systems and networks, the
courses provide the practical and up-to-date experience required by
the major IT companies that develop computer systems and networks.
