\section{Vision om D-program\-mets framtid}

\meta{Patrik: Detta är delar av den text jag skrev i ansökan för att
  bli PA för Datateknik november 2010.}
%
% Vi lever i en oerhört spännande tid när datatekniken infiltrerar
% snart sagt varje del av samhället.
% %
% Vi är alla användare av datatekniska produkter och tjänster och det
% är svårt att tänka sig vårt moderna samhälle utan mobiltelefoner,
% Internet och ABS-bromsar.
% %
% För användare räcker det att tekniken fungerar (och ser bra ut) men i
% bakgrunden måste någon planera, utveckla, implementera och underhålla
% (de ofta programvarubaserade) systemen.
% %
% Det är där som huvudfokus bör vara för D-programmet nu och i framtiden.
% %
% Samhället (inklusive industrin) behöver dataingenjörer som har koll på
% god kvalitet, hög produktivitet och låga kostnader.
% %
%
% Men en högre utbildning är så mycket mer än så---personlig utveckling,
% nyfikenhet och innovation är också viktiga komponenter.
% %
% Många kommer inte att jobba just med det de lärt sig på Chalmers men
% en gedigen bakgrund i teknik och problemlösning kan räcka långt i
% nästan alla områden.
% %
% Många går vidare till forskning och utveckling och för dem gäller det
% att tidigt hitta utmaningar och ingångar som leder dem vidare.
%
% Slutligen finns också övergripande utmaningar för mänskligheten i stort
% som delproblem inom den globala omställningen till en hållbar
% utveckling.
% %
% Här känner jag att det finns mycket att göra med datateknik som
% verktyg och därmed bör det finnas goda chanser att utveckla
% utbildningen så att detta framgår redan tidigt.
% %

Några till punkter --- jag vill
\begin{itemize}
\item utnyttja studenternas egna erfarenheter och intressen genom att
  erbjuda rejäla möjligheter till valfrihet redan inom de första tre
  åren.
\item se ett fortsatt nära samarbete med IT och med datavetenskap (på
  GU), men även med E, Z, samt SE\&M (på GU).
\item arbeta i nära samråd med berörda institutioner och lärare.
\item främja datorstödd modellering och högnivåspråk inom alla ämnen:
  datavetenskap, datorteknik, mate\-matik, teknikbredd, \ldots
\item utveckla kandidatdelen av programmet baserat på vad
  masterutbildningen behöver (enl. lärare och studenter).
\item utveckla masterprogrammen baserat på vad framtida arbetsgivare
  (industri, akademi, samhälle) behöver och studenter vill.
\item öka rekryteringen av studenter inom EU --- särskilt från
  östeuropa.
\end{itemize}
